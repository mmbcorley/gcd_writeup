\documentclass[a4paper,man,natbib]{apa6}
\usepackage{microtype}
\usepackage{mathtools} % needed
\usepackage{hyperref}
\usepackage{tabularx}
\usepackage{lingex}
\usepackage[modulo,displaymath,pagewise]{lineno}

\newcolumntype{Y}{>{\raggedright\arraybackslash}X}
\usepackage[normalem]{ulem}
\hypersetup{hidelinks=True}
\newcommand*{\smex}[1]{\textit{#1}} % 'small example'
\newcommand*{\spex}[1]{``{#1}''} % 'spoken example'
\newcommand*{\term}[1]{\emph{#1}} % introducing a new term
\newcommand*{\citegen}[1]{\citeauthor{#1}'s~(\citeyear{#1})}
\newcommand*{\SE}{\mathit{SE}} % fix funny "SE" spacing
\newcommand{\resultsLog}[3]{$\beta = #1$, $\textnormal{SE} = #2$, $p #3$}
\newcommand{\resultsLM}[3]{$\beta = #1$, $\textnormal{SE} = #2$, $t #3$}

\title{Such gesture, very lie, wow}
\author{reorder(MC,JK,JL)}
\affiliation{Psychology, PPLS, University of Edinburgh}
\ifapamodeman{\note{\begin{flushleft}%
Josiah King\\
Philosophy, Psychology and Language Sciences\\
University of Edinburgh\\
7~George Square\\
Edinburgh EH8~9JZ, UK\\[1ex]
\url{J.P.J.King@sms.ed.ac.uk}
\end{flushleft}}}

\abstract{
Previous research suggests that, when questioning the veracity of an utterance, we perceive certain non-linguistic behaviours to indicate that a speaker is being deceptive.
Recently, work has highlighted how listeners' associations between speech disfluency and dishonesty happen at the earliest stages of reference comprehension, suggesting that contextual information about the manner of spoken delivery influences pragmatic judgments simultaneously with the processing of lexical information.
The studies presented here investigate the integration of visual information about a speaker into judgments of deception.
By studying the time-course of judgments of a speaker's (dis)honesty when presented with different visual cues to deception, we ask whether listeners are relying upon a rule-of-thumb association between visual cues and deception, or whether the link between gestures and perceived deception requires a more complex inferential process.
Participants saw and heard a video of a potentially dishonest speaker describe treasure being hidden behind a named object, while also viewing both the named object and a distractor object. 
Their task was to click on the object behind which they believed the treasure to actually be hidden.
Eye- and mouse-movements were recorded. 
Experiment 1 investigates the time-course of listeners' associations between visual cues and deception, using a variety of static and dynamic cues.
Experiment 2 focuses on a dishonesty-bias for adaptor (fidgetting) gestures, and asks whether listeners' judgments of deception correlate with how nervous they think the speaker appears in each video.
Results show that the visual modality can have a rapid and direct influence on pragmatic judgments, supporting the idea that communication is fundamentally multimodal, and should be studied as such.
Although the nonverbal delivery of an utterance is found to influence the early stages of reference comprehension, establishing deception-judgments based on visual cues appear to be more gradual than previous studies have suggested it is for spoken cues.
}


\begin{document}


%JK > exact T vals.
%JK > mouse track discuss



\shorttitle{What do liars look like?}
\maketitle
\linenumbers
\noindent
%intro (%JK too long?)
That people deceive one another is an inescapable aspect of everyday communication.
After studying participants' social interactions over a one week period, \citet{DePaulo1996} concluded that people lie on average twice a day.
The idea that it possible to detect deceit --- that there are systematic differences in people's behaviour depending on whether they are telling a truth or a lie --- has long captivated human interest in a variety of areas, from criminal interrogations to the business world.\footnote{Where, in some cases, ``how to lie'' is equally as important. See https://www.marketplace.org/2008/02/18/busness/lying-essential-doing-business}
Much research has investigated the behaviours that listeners associate with lying, covering a wide range of multi-modal phenomena from speech disfluencies and vocal pitch to visual cues such as postural shifts and hand movements.
Few studies, however, have studied the time-course of inferences to deception, and fewer still have done so in a multi-modal setting.
The present experiments investigate listeners' associations of a speaker's nonverbal behaviour with the veracity of an utterance, studying how and when visual information about a speaker is integrated into pragmatic interpretations of speech.

%could actually start here?
In most natural communication, speakers can convey information via multiple channels.
Along with spoken delivery, a speaker's gestures, postures and facial expressions, all offer information which can have a bearing on the non-literal---or pragmatic---interpretation of a message, for instance via conveying emotion \citep{Busso2004, Gregersen2005}
One such pragmatic interpretation of a spoken utterance is the judgment of a statement's veracity.
Research suggests that there are many aspects of nonverbal delivery which we, as listeners, believe to be \term{cues-to-deception}. 
In an analysis of 33 studies, \citet{Zuckerman1981} found that nine out of the ten visual cues-to-deception that were included in the analysis were believed to be indicative of deceit. 
In a further subset of 13 studies reporting relationships between cues and subsequent deception judgments (rather than explicit beliefs about cues), three (smiling, gaze, and postural shifts) of the four available visual cues were associated with perceived dishonesty.

Interestingly, listeners appear to hold these associations and beliefs despite the reliability of cues as actual signals of deception being inconclusive: \citet{Zuckerman1981} found only two visual cues (shrugs and fidgetting) be associated with actual deceit, and a  more recent meta-analyses found little evidence of a relationship between lying and almost all forms of movement \citet{DePaulo2003}.
. %% Jia: another meta-analysis to cite here is Hartwig (2011) about cues to actual deception being weak (https://www.researchgate.net/profile/Maria_Hartwig/publication/51251661_Why_Do_Lie-Catchers_Fail_A_Lens_Model_Meta-Analysis_of_Human_Lie_Judgments/links/02bfe5109718134014000000/Why-Do-Lie-Catchers-Fail-A-Lens-Model-Meta-Analysis-of-Human-Lie-Judgments.pdf) - courtesy of the reviewer 2 who thinks my dialogue study is crap and implies that we should rely more on meta-analyses rather than individual studies...
Furthermore, some studies have found deception to be inversely related to the cues we think of as deceitful: lying has been linked with a \emph{decrease} in hand, arm, and leg movements \citep[e.g.][]{DePaulo1992, Ekman1989, Vrij1995}, as well as a reduction in illustrative gesturing \citep[e.g.][]{DePaulo2003, Cohen2010}.
Even speakers' post-hoc perceptions of their own gestures when lying have been found to be at odds with how they actually behaved:
\citet{Vrij1996} found that after partaking in two interviews --- one in which they were truthful, the other dishonest --- participants believed that their movements increased when lying, even though a decrease actually occurred.
%(whether or not participants were informed before-hand that deception is usually associated with a decrease in movements had no effect).

We tend to hold misguided judgements about which behaviours really co-occur with deception (see \citet{Vrij2000}), suggesting that the links which listeners' draw between visual information and deception are likely not a result of learning these associations from experience.
What is more, this is reinforced by the fact that in everyday communication, there are few occurrences where listeners are given immediate feedback on the honesty of a given utterance.
Why then do listeners' hold such strong associations and beliefs between visual cues and lying?
One suggestion is that this association is based on a set of cues which listeners consistently associate with deception via a rule-of-thumb based heuristic \citep{DePaulo1982}.
This set of cues may be predicated on lay beliefs about deception, or on introspection as a speaker rather than experience as a listener, but ultimately leads to the somewhat innaccurate model of what deceitful behaviour looks and sounds like. 

Recent research by \citet{Loy2017} investigating the time-course of the association between lying and speech disfluency supports the idea that listeners might rely on such a rule-of-thumb based approach when judging deception based on spoken delivery
\citet{Loy2017} used a visual world paradigm in which participants were presented with two objects along with utterances describing the location of some treasure purportedly hidden behind one of the objects.
These utterances were presented as having been elicited in a previous experiment, in which the speaker was said to have been lying some of the time.
Crucially, \citet{Loy2017} manipulated the manner of spoken delivery, with half of the experimental items containing a speech disfluency.
Participants were tasked with choosing where they \textit{believed} the treasure to really be hidden --- the object named in the utterance (indicating a judgment of honesty), or a distractor (indicating dishonesty).
Participants were more likely to judge disfluent utterances as dishonest than fluent ones (as indicated by more clicks on the distractor on disfluent trials). 
Importantly, \citeauthor{Loy2017}'s results showed an early bias in both eye and mouse movements towards the not-referred-to object, suggesting that speech disfluency is already incorporated into listeners' idea of deceptive speech, having an immediate effect on their interpretation of the utterance. 

Little is known, however, about whether such rapid integration of cues to deception might extend to the visual channel of communication. 
Adhering to a rule-of-thumb association may be feasible for spoken cues, and evidence suggests that some effects of speech interruptions on comprehension do not discern between a spoken \spex{um} and an artificial tone \citep{Corley2011}.
Co-speech movements, however, are substantially more varied than speech interruptions, serving as both potential markers of metacognitive states and planning processes, and as an alternative modality in which a speaker can convey semantic information (such as via an illustrator).
Any rule-of-thumb heuristic between visual information and deception must be refined enough to discern between the type/content of co-speech movements, or risk potentially over-attributing any cue as a sign of deceit.
Furthermore, listeners also associate static visual cues with deception \citep[e.g. eye-gaze,][]{Zuckerman1981a}, suggesting that judgments of dishonesty are not linked just to variations in movement, but to a wide array of visual cues. 
It is not clear how such a heuristic might map such varied visual information to deceit.

An alternative account to the rule-of-thumb explanation suggests that non-linguistic information might influence listeners' judgments of deception via a form of on-the-go speaker-modelling: 
When presented with a possibly deceitful utterance, listeners might be inferring information about a speaker's metacognitive states, linking visual and spoken cues to deception via the perception of, e.g. nervousness, or cognitive effort. 
Such an account requires more complex inferential processing, being a two-step inferential process linking gesture to, e.g. anxiety, which is in turn attributed to the intention to deceive.
%% Jia: also I'm not sure whether it's best to call it a form of speaker-modelling here or not - it is speaker modelling but it doesn't seem to quite contrast with the rule-of-thumb account (wouldn't introspection as a speaker, which you mention earlier, create some sort of a speaker model as well?). I might be overthinking things, but it's worth thinking about whether this could be called something more specific (or see if Martin cares...)
%JK link car-horns paper in here?

To date, the link between visual cues and perceived deceit has been studied only in terms of after-the-fact judgments, or assessing listeners' explicit beliefs about cue validity (see \citealt{Vrij1996a, Zuckerman1981a}).
Studying the time-course of judgments about the honesty of an utterance which is accompanied by different visual cues can help to shed light on how these cues are integrated.
We extend the `treasure game' paradigm from \citet{Loy2017} to include a video of the potentially deceptive speaker describing the location (behind one of two objects) of some hidden treasure on the screen while listeners attempt to guess the true location based on whether they believe the speaker to be lying or telling the truth. 
%%% Jia: we should include a figure of the display here %%%
Crucially, we manipulate the presence or absence of potential visual cues to deception in the video.
Experiment 1 investigates how the time-course of deception-judgments varies between different types of cues, including both increases in movement and different static postures.
If listeners associate visual cues with deception via a rule-of-thumb association, then a bias towards the not-referred-to object should occur early on following trials presenting visual cues to deception.
In Experiment 2 we explore the possibile role of perceived anxiety as an intermediary step in judgments about a speaker's honesty.
Concentrating on adaptor gestures (fidgeting movements), having been previously linked with anxiety \citep{Gregersen2005}, we ask whether listeners' judgments of deception are driven by their perceptions of how nervous they think the speaker to be when producing these gestures.

\section{Experiment 1}
Experiment 1 makes use of eye- and mouse-tracking to investigate the time course of listeners' judgments about the honesty of an utterance, and how these judgments are influenced by the occurrence of various visual cues. 
The experiment was presented as a `lie detection game', with each trial presenting a video and audio recording of a potentially deceptive speaker describing the location of some hidden treasure on the screen (behind one of two possible objects). 
Participants were tasked with clicking on where they believe the treasure to be hidden based on their judgement of the speaker's honesty.
We employed 3 types of visual cues --- trunk movements, adaptor gestures, and different postures --- investigating if and how listeners associated any of these with deception.
%% (add) Our results show that...


\subsection{Materials}
Visual stimuli consisted of the same 120 line drawings from \citet{Snodgrass1980} which were used in \citet{Loy2017}, sixty of which served as referents and the other sixty distractors.
Referents were randomly paired with distractors and presented in across sixty trials. 
In each trial, the speaker named one object (referent) as that which concealed the treasure; the other object is hereafter named the distractor.
Each referent was associated with a recording specifying the image as the object that the treasure was hidden behind (``The treasure is behind the <referent>'').
Along with these images, each trial presented a video (with no audio) of a person who was purported to be the speaker of the utterances. 
So that videos could be counterbalanced across referents, and thus across utterances, the face of the person in the video was blurred. 
This meant that, when presented with a given utterance, it was believable that both audio and visual stimuli had been produced concurrently. 

Sixty videos (30 Cue, 30 No-Cue) were created. 
The thirty no-cue videos were comprised of ten recordings (each presented thrice) of a speaker sitting motionless with her hands on either side of a tablet on a table, upon which the referent, distractor, and location of treasure was purported to be displayed.
Ten videos presented the speaker making one of five trunk movements, ten presented various adaptor gestures (e.g. finger tapping, head tilts), and ten presented the speaker sitting motionless but in a different posture to that of the no-cue videos (e.g. hand on chin, arms crossed, etc).

%JK essentially, we've got utterance initial (trunk movmements); utterance medial (adaptor gestures) and utterance global (posture)
A variable video-to-speech-onset was adopted for the 30 cue videos, controlling the position of the cues relative to speech. 
%%Jia: sorry, I can't fully remember what we did at this point - I thought it was gesture-to-speech onset (rather than video to speech)? 
%JK it was video-to-speech - basically we tried to audio and video to seem believable. i don't think we measured when exactly the cues began/stopped other than the trunk movements.
For trunk movements, the frame at which the movement ended was identified and used as the point of utterance onset (Mean onset = 1430ms, SD = 440ms).
To control for any sensitivity to the duration of pre-utterance video (which could be interpreted as speech-initiation time, in turn a potential cue to deception), these durations were matched in the no-cue videos.
For adaptor gestures, the movements overlapped with speech, with the duration of overlap determined individually for each video according to what looked most believable to the researchers (Mean = 1370ms, SD = 410ms). 
%JK is this speech gesture overlap, or gesture_onset? or speech onset? check and rewrite
These durations were matched in the videos that presented the speaker sitting in a different posture.

As with \citet{Loy2017}, 20 critical referents were counterbalanced across two lists. 
Each list contained 10 cue and 10 no-cue videos.
The 10 cue videos were trunk movements since we predicted these to be the most likely visual cue to elicit judgements of deception\citep{Vrij1996a}
%Jia: think we should try and provide a cite for this (one of the Vrij studies?)
% JK: have done, but it's not great. Vrij splits "trunk movements" (moderate assocation) and "postural shifts" (strong association). I think we used trunk movements to mean postural movements. It's not clear what Vrij uses. 
%% Jia: hm fair enough, I thought it would be relatively straightforward to find a useful cite, but I guess Vrij can be quite unclear. Let's leave it for now and I'll see if I come across a more useful one at all
The remaining 40 referents were randomly paired with one of the remaining videos (10 adaptors, 10 different postures, 20 no-cue) for each participant, with no repetition of referents across videos.

\subsection{Procedure}
Stimuli were displayed on a 21~in.\@ CRT monitor, placed 850~mm from an Eyelink~1000 Tower-mounted eye-tracker which tracked eye movements at 500~Hz (right eye only). 
Audio was presented in stereo from speakers on either side of the monitor. 
Mouse coordinates were sampled at the frame rate of the videos (25~fps, or every 40~ms).
The experiment was presented using OpenSesame version~3.1 \citep{Mathot2012}.
Eye movements, mouse coordinates and object clicked (referent or distractor) were recorded for each trial.

Figure \ref{fig:v1_trial} presents a sample trial from the experiment. 
\begin{figure}[Ht]
  \centering
	\includegraphics[width=\linewidth]{./img/e7_trial.png}
  \caption{Procedure of a given trial, Experiment 1}
  \label{fig:v1_trial}
\end{figure}
Between trials, participants underwent a manual drift correct to ensure accurate recordings from the eye-tracker.
After this the fixation dot turned red for 500ms. 
This was replaced by the two objects (referent and distractor), which were displayed on the screen for 2000ms.
The video then appeared and the cursor was centred and made visible.
Playback of the utterance began after the variable speech onset associated with each video (Mean = 1410ms, SD = 410 ms). 

The instructions emphasised that the videos participants saw were recorded from a previous experiment, in which the speaker had to describe the location of some hidden treasure with the aim of misleading the listener into choosing the wrong location.
Participants were instructed to click on the object behind which \textit{they believed} the treasure to be hidden, with the overall aim of accumulating as much treasure as they could across the experiment.
Participants received no feedback after their object clicks, except on bonus trials, which are described in the next section.

Participants completed five practice trials (one of which was presented as a bonus round) prior to the main experiment. 
Two of these included no cue, two displayed the speaker in different postures, and one displayed the speaker making a trunk movement.

\subsection{Bonus Rounds}
To maintain motivation throughout the study, participants were told that there were a number of ``hidden bonus rounds'' which offered more treasure than regular rounds.
25\% of trials (half in the cue condition; half in the no-cue condition) were randomly designated as bonus rounds for each participant.
These trials were visually identical to regular trials, with the exception of a message informing participants that they had successfully located bonus treasure following their mouse click (regardless of the object chosen).

Participants were also told that the top scorers would be able to enter their names on a high-score table, which was shown at the beginning of the experiment. 

\subsection{Post-test Questionnaire}
Participants were asked to complete a short post-test questionnaire. 
The questionnaire contained three questions, the most important of which asked if participants noticed anything odd about the visual or audio stimuli.
Any participant who indicated that they noticed anything unusual was then verbally questioned, to decide whether they believed that the speech and gesture had been produced naturally and simultaneously.
All participants were subsequently debriefed, during which they were told that the audio and video were created separately and stitched together, and asked again verbally if they noticed anything unusual in that respect. 
Their responses to the questionnaire and debrief were used as exclusion criteria for the analysis.

\section{Results}
Twenty-four native English speaking participants took part in the experiment for a planned sample size of twenty.
Participants were recruited from the University of Edinburgh community, and participated in return for a payment of \pounds{}4.
Data from four participants who indicated suspicion of the proposed origins of the audiovisual stimuli based on the post-test questionnaire and/or debrief were removed from all analyses.

\subsection{Analysis}
Analysis was carried out in R version~3.4.4 \citep{Rbase2017}, using the lme4 package \citep{Bates2015}. 
Trials in which participants did not click on either the referent or distractor (0.003\% of trials) were excluded from all analyses. 

Object clicked (referent or distractor) was modeled using mixed effects logistic regression, with fixed effects of cue type (No-Cue, Different Postures, Trunk Movements, Adaptor Gestures) and video-to-speech duration (Z-scored), thus controlling for any possible effect of perceived speech latency on judgments of dishonesty.
Cue type was dummy coded with No-Cue as the reference level.
Random intercepts and slopes for cue type and video-to-speech were included by-participant, along with random intercepts by-referent.
Reaction times (measured from referent onset) were modelled with the same fixed and random effect structure.
Following \citet{Lo2015}, we compared mixed effects logistic regression models which specified an identity link function, assuming gaussian, gamma and inverse gaussian distributions.

In previous studies using the treasure game paradigm, eye- and mouse- movements have been analysed over the time window starting at the onset of the referent name and extending for 800~ms; just beyond the duration of the longest critical referent (776~ms). 
Because the current study includes in the analysis all available referents (rather than the subset used in previous experiments), this window is extended to 0--1100~ms to include the duration of the longest referent (1062~ms).

Eye fixation data was averaged into 20~ms bins (of 10 samples) prior to analysis.
For each bin, we calculated the proportions of time spent fixating the referent or the distractor, resulting in a measure of the proportions of fixations on either object over time.

The position of the mouse was sampled every 40~ms.
Using the $X$ coordinates only, we calculated the number of screen pixels moved and the direction of movement (towards either referent or distractor).
We then calculated the cumulative distance travelled towards each object over time as a proportion of the cumulative distance travelled in both directions up until that time bin.
Movements beyond the outer edge of either object were considered to be `overshooting' and were not included in calculations (0.8\% of samples).
Eye- and mouse- biases were calculated from the proportions of referent to distractor fixations, and were subsequently empirical logit transformed \citep{Barr2008}. 
In these measures, a value of zero indicates no bias towards either object, and positive and negative values indicate a bias towards the referent and distractor respectively.

Eye and mouse data was modelled over the time window from 0 to 1100~ms post-referent onset using linear mixed effects models, with fixed effects of time, cue type, and their interaction.
Random intercepts and slopes for time were included both by-referent and by-participant, along with by-participant random effects of cue type.
Following \citet{Baayen2008}, we considered effects in these models to be significant where $|t|>2$.

%JK Maybe just get rid of this analysis? >>
%% Jia: coming back to it after reading the results: I think it's fine, since it supports the later-effects conclusions, but probably needs to be explained/justified a bit more - for instance why do that over just extending the window of analysis for the elogit modelling?
%%JK yeah, this isn't massively clear because I think we should decide between one or the other.
%%they're both on the elogit ref/dis bias. the below is basically lmm with polynomials (so captures curvature). 
As visual inspection of the time-course of fixations towards either object suggested that there was a later effect of visual cues in participants' decision of which object to click on, growth curve analysis (see \citealt{Mirman2008}) on the empirical logit transformed referent-to-distractor bias was used to investigate this further. 
This analysis was conducted over the time window from 0 to 1815~ms post-referent onset (average time-to-click), and included 3 degrees of orthogonal polynomials for time, based on the pattern of the elogit referent-distractor bias having 2 turning points. 
These time polynomials, along with their interaction with cue type, were included as fixed effects in a linear mixed effects model, with random intercepts and slopes for all time polynomials both by-participant and by-referent.


\subsection{Object clicks} 
Across the experiment, participants clicked on the referent in 55\% of trials and the distractor in only 45\%.
Table \ref{table:v1_clicks} shows the percentage of clicks across all participants to either object following different types of visual cue.
When presented with an utterance accompanied by no-cue, participants showed a bias toward a final interpretation of the utterance as truthful, with more clicks to the referent than the distractor \resultsLog{0.62}{0.16}{<0.001}.
Analysis showed that all visual cues resulted in a reduction of this bias, with adaptor gestures (\resultsLog{-1.03}{0.34}{=0.003}) showing a greater change than different postures and trunk movements (\resultsLog{-0.72}{0.31}{=0.017} and \resultsLog{-0.62}{0.26}{=0.018} respectively). 
The duration of video shown prior to the beginning of speech was not found to be associated with which object was eventually clicked.
%Comparisons --- via both AIC and BIC --- of reaction time models suggested that an inverse gaussian distribution provided the best fit to the observed data. 
Neither cue type nor duration of video prior to speech was associated with a significant change in reaction times.

\begin{table}
\caption{Breakdown of mouse clicks recorded on each object (referent or distractor) by type of visual cue for Experiment 1}
\label{table:v1_clicks}
\begin{tabularx}{\linewidth}{YYYYY}
\hline
& No-Cue & Different Posture & Trunk Movement & Adaptor Gesture \\
Clicks to Referent & 63.8\% & 48.0\% & 49.7\% & 41.5\%  \\ 
Clicks to Distractor & 36.2\% & 52.0\% & 50.3\% & 58.5\% \\
\hline
\end{tabularx}
\end{table}

\subsection{Eye movements}
Figure \ref{fig:v1_eye} shows the time-course of fixations to referents and distractors over 2000~ms from referent onset, split by each type of video.

Analyses conducted over the period from referent onset to 1100~ms post-onset (duration of the longest referent) showed that, when presented with a no-cue video, participants displayed a fixation bias towards the referent which increased over time (\resultsLM{1.15}{0.29}{=3.99}).
For videos which presented the speaker either in a different posture, or producing an adaptor gesture, this increasing bias to the referent was significantly reduced
(\resultsLM{-0.97}{0.13}{=-7.37}, \resultsLM{-0.59}{0.13}{=-4.50}), but this was not the case for videos of trunk movements (\resultsLM{-0.25}{0.15}{=-1.67}). 

Figure \ref{fig:v1_gca} shows the time-course of the elogit referent-distractor bias, alongside the fitted values from the growth curve model. 
Growth curve analysis over the period from referent onset to 1815~ms post-onset (mean click time) showed that when presented with a no-cue video, participants showed a tendency to fixate on the referent over the course of this time period (\resultsLM{0.67}{0.11}{=6.03}). 
Significant effects of cue type on the intercept term indicated lower overall referent fixations during this period after viewing a different posture (\resultsLM{-0.36}{0.04}{=-10.10}), a trunk movement (\resultsLM{-0.29}{0.4}{=-7.36}), or an adaptor gesture (\resultsLM{-0.67}{0.4}{>=-18.89}), in comparison to the no-cue videos.
Significant interactions of cue type and the linear time coefficient indicate that this reduction in referent-bias dependent on cue type increases over the course of the window, with adaptor gestures showing a larger reduction relative to no-cue videos (\resultsLM{-116.89}{11.62}{=-10.06}) than different postures (\resultsLM{-75.41}{11.65}{=-6.47}) and trunk movements (\resultsLM{-58.78}{12.97}{=-4.53}). 
% JK: Beyond this, different posture videos showed a more gradual initial tendency towards the referent relative to the no gesture videos (\resultsLM{39.83}{11.64}{>2}) and trunk movement videos showed an increased ?? -- the little curve at the end.. (\resultsLM{33.50}{12.75}{>2}).


\begin{figure}[Ht]
  \centering
	\includegraphics[width=\linewidth]{./img/e7_fixations.pdf}
  \caption{Eye-tracking results for Experiment 1: Proportion of fixations to each object (referent or distractor) and the video, from 0 to 2000 ms post-referent onset, calculated out of the total sum of fixations for each 20~ms time bin. Shaded areas represent $\pm$ 1 standard error of the mean.}
  \label{fig:v1_eye}
\end{figure}

\begin{figure}[Ht]
  \centering
	\includegraphics[width=\linewidth]{./img/e7_gcamodel.pdf}
  \caption{Elogit referent-distractor bias, and fitted values from growth curve analysis}
  \label{fig:v1_gca}
\end{figure}


\subsection{Mouse movements}
Figure \ref{fig:v1_mouse} shows the time-course of the proportions of cumulative distance the mouse moved towards the referent and distractor for 2000~ms from referent onset, split by each type of video.
Analysis on the time window from 0 to 1100~ms post-referent onset patterned with the eye-tracking data:
When viewing a speaker making no visual cue, participants showed an increasing tendency to move towards the referent over this period (\resultsLM{1.08}{0.22}{=4.86}).
As with the eye-movements, different postures and adaptor gestures resulted in a weakening of this referent-bias (\resultsLM{-0.81}{0.13}{=-6.28} and \resultsLM{-0.77}{0.13}{=-5.87} respectively), but trunk movements did not (\resultsLM{-0.10}{0.14}{=-0.69}). 

\begin{figure}[Ht]
  \centering
	\includegraphics[width=\linewidth]{./img/e7_mouset.pdf}
  \caption{Mouse-tracking results for Experiment 2: Proportion of cumulative distance traveled toward each object from 0 to 2000 ms post-referent onset. Proportions were calculated from the total cumulative distance participants moved the mouse until that time bin (from video onset, when cursor was made visible). Shaded areas represent $\pm$ 1 standard error of the mean.}
  \label{fig:v1_mouse}
\end{figure}


\section{Discussion}
Experiment 1 investigated how the pragmatic inferences listeners make about a speaker's honesty are influenced by the presence of different types of movements and postures. 
We measured eye- and mouse- movements of participants who were presented with a task in which they made decisions about the true location of some treasure based on audio and video of a potentially deceptive speaker making a statement about the treasure's location.
Participants were thus making implicit decisions about the honesty of each utterance.
Our findings suggest that a speaker's nonverbal behaviour influences listeners' judgments about (dis)honesty.

As in previous studies using versions of this paradigm \citep{Loy2017, King2018}, participants showed a tendency to interpret an utterance as truthful when it had been presented without any potential cue to deceit.
Utterances presented alongside any cue --- both movement and different postures --- weakened this tendency, supporting previous findings in the deception literature that listeners' final judgments of deception are influenced by a speaker's nonverbal behaviour.
However, only the presence of adaptor gesturing resulted in significantly more final judgments of deception than truthfulness (as indicated by more clicks to the distractor over the referent).

The influence of the visual channel was also evident in participants' early stages of utterance processing. 
Across videos both with and without a cue, participants showed an initial tendency to fixate and move the mouse toward the referent over the distractor; however, videos of different postures and adaptor gestures weakened this referent bias.
Interestingly, this was not the case for videos of the speaker producing a trunk movement, despite the movement having been presented at an earlier point---relative to speech---than the adaptor gestures.
One possible explanation for this could be due to the fact that at the point of referent onset, the audiovisual information immediately available to the listener was comparable to the no-cue videos, since the trunk movements would have ended by that point.

This explanation, however, is at odds with the bias towards judgements of deception following utterance-initial disfluencies observed in previous work \citetext{\citet{Loy2017}, Experiment 1}.
We cannot at present reconcile this discrepancy, although we suggest that they may point toward differences in the mechanisms underlying the integration of audio and visual cues with a pragmatic inference.
Future research could explore the simultaneous integration of audio and visual information during judgements of deception, comparing the relative time course with which the two are integrated by the comprehension system.

Despite this evidence of an early effect of visual cues on reference comprehension, the point at which listeners' preferences toward the referent tended to be matched (or overtaken, in the case of adaptor gestures) by a preference for the distractor appeared at approximately 1000~ms post-referent onset. 
In other words, the distractor-bias following visual cues to deception appeared relatively late in processing.
This contasts with previous studies using audio-only versions of the `treasure-game' paradigm: Participants in \citet{Loy2017} showed little to no initial referent-bias following a disfluent utterance, with a distractor-bias appearing approximately 600~ms post-referent onset following both utterance-initial and utterance-medial disfluencies. 
This difference could be a result of visual information in the videos detracting participants from fixating the two objects. 
In particular, the integration of visual information into a judgement of deception could be delayed relative to that with a judgement of truthfulness: 
This would explain why listeners' distractor-bias in the cue conditions was delayed, despite the fact that their initial referent-bias across all conditions emerged early on at approximately 300~ms post-referent onset --- a point comparable to previous studies (\citealt{Loy2017, King2018}). 

Judgements of deception took substantially longer to unfold than in previous studies using the same paradigm (minus the video component), suggesting that listeners might be drawing links between visual cues and dishonesty via a more cognitively demanding process than a straightforward heuristic.
Although visual information was found to influence the early stages of comprehension, it is unclear whether this should be taken as evidence of emerging judgments of deception, or merely as differences in visual saliency between experimental stimuli (e.g., a trunk movement vs. an adaptor).

To investigate further the possibility that the association between visual cues and deception passes via some sort of modelling of the speaker's cognitive state (e.g. nervousness), we conducted Experiment 2.

\section{Experiment 2}
To explore whether associations between visual cues and judgements of deception are driven by perceived nervousness in the speaker, Experiment 2 focussed on movements which are associated with anxiety (see \citet{Gregersen2005}).
To acheive this we included a task after the eye-tracking segment asking participants to rate each video for how nervous they thought the speaker to be, and investigated whether these ratings parallelled with judgements of deception.
We used a selection of adaptor gestures which are associated with nervousness based on \citet{Gregersen2005}, which were pre-tested for perceived anxiety in the speaker. 
If adaptor gestures are linked to deception via perceived nervousness, then gestures which were participants rated as more nervous should be associated with more judgements of dishonesty. 

Using the same paradigm as Experiment 1, participants in Experiment 2 heard utterances accompanied by a video of a speaker either producing an adaptor gesture or sitting motionless, and were tasked with making an implicit judgement on whether the speaker was lying or telling the truth. 
After the task, participants were asked to rate each video (without audio) on how nervous the speaker looked.
This provided us with a measure of association between the speaker's body language and their perceived nervousness.

\subsection{Materials}
A subset of 40 images (20 referents; 20 distractors) from those in Experiment 1 were used across twenty trials.
As in Experiment 1, these images were displayed in referent-distractor pairs, with each pair shown alongside a recorded utterance naming the referent as the location of the treasure.
Following \citet{Loy2017}, we used the same set of referents and distractors which had been matched for both ease of naming and familiarity.

As in Experiment 1, each pair of images and recorded utterance was presented alongside a video clip of a person purported to be the speaker of the utterance.
Twenty video clips (10 adaptor gestures; 10 no-cue) were used. 
Based on participant feedback from Experiment 1 (in which several participants mentioned basing judgments on how relaxed the speaker's posture appeared to be), care was taken to ensure that the no-cue videos presented the speaker in a relaxed posture. 
Adaptor gestures were based on descriptions of anxious nonverbal behaviour from \citet{Gregersen2005}.
Videos were chosen from a pre-test which asked participants to rate 28 silent videos (18 different adaptor gestures; 10 no-cue) for their perceived nervousness of the speaker. 
10 native english speakers were told that they were going to watch videos (without audio) of someone being questioned in a stressful situation, and were asked to rate how nervous the speaker looked in each video (1: very relaxed, 7: very nervous). 
The 10 adaptor gestures with the highest ratings (Mean = 4.1, SD = 1.5) were included in the experiment, along with the 10 no-cue videos (Mean = 1.9, SD = 1.1).
%JK the way I'm phrasing this experiment, we really should have used gestures with >variance in pre-test ratings?

The 20 referents were counterbalanced across two lists such that each referent that occurred with a gesture video in the first list occurred with a no-cue video in the second.
The pairings of referents with specific videos/gestures within each condition was randomised on each run of the experiment.

\subsection{Procedure}
The experiment procedure was identical to that of Experiment 1 with two minor changes.
First, speech initiation time (the duration between video playback and utterance onset) was set a fixed constant of 1170~ms after the beginning of the video on each trial.
Second, since the experiment was considerably shorter, we did not include any `bonus' trials which displayed a message stating that treasure had been found after an object click.
Hence, participants did not receive any feedback in any of the trials in this experiment.

After the main task, participants were asked to watch all 20 videos again, without audio, and asked to rate how nervous they thought the speaker looked (using the same 1-7 scale as described above).
Participants then completed the same post-test questionnaire as in Experiment 1, with data being excluded from analysis based on the same criteria.

\section{Results}
Twenty-three native English speaking participants took part in exchange for \pounds{}3 compensation. 
Data from three participants were excluded due to suspicion of the audiovisual stimuli being scripted (based on the post-test questionnaire and questioning during debrief), hence the final dataset included data from 20 participants.

\subsection{Analysis}
We followed the same analysis strategy as was used for Experiment 1, with the experimental manipulation of cue presence being a dichotomous Cue vs.\@ No Cue.
Trials which did not result in a click to either object (0.8\%) were excluded from analyses.

Analyses of object clicks and reaction times did not control for video-to-speech duration, since this was controlled in the experimental design.
To investigate whether listeners' deception judgments were predicted by their subsequent ratings of perceived nervousness, analysis of object clicks included participants' post-test ratings (Z scored) of each video as a fixed effect, along with its interaction with presence of cue.

The time window of analysis for eye- and mouse- movements was reduced to the 800~ms following referent onset, based on the fact that the subset of referents included in this experiment had a maximum duration of 776~ms. 
For the mouse movement analysis, movements beyond the outer edge of either object were excluded from analyses (1\% of samples).
Because Experiment 2 fully counterbalanced cue presence across all referents, random effects of cue presence and time and their interaction were included both by-referent and by-participant.

\subsection{Object clicks}
Across the experiment, participants clicked on the referent in 53\% of trials and the distractor in 47\%.
Table \ref{table:v2_clicks} shows the proportions of clicks to either object following videos displaying either a cue (adaptor gesture) or no cue.
As in Experiment 1, participants showed a bias toward a final interpretation of the utterances without visual cues as truthful, with more clicks to the referent than the distractor \resultsLog{1.66}{0.56}{=0.003}.
Patterning with the adaptor gestures used in Experiment 1, utterances presented with an adaptor gesture here resulted in bias towards participants clicking the distractor (\resultsLog{-2.99}{0.69}{<0.001}).
Participants post-test ratings of how nervous they perceived the speaker to be in each video did not predict which object they clicked, for either videos presenting gestures (\resultsLog{-0.10}{0.58}{=0.87}) or videos presenting no cue (\resultsLog{0.23}{0.62}{=0.71}).
There was no effect of cue presence on time to click.

\begin{table}
\caption{Breakdown of mouse clicks recorded on each object (referent or distractor) by presence of cue for Experiment 2}
\label{table:v2_clicks}
\begin{tabularx}{\linewidth}{YYY}
\hline
& No Cue & Cue (adaptor gesture) \\
Clicks to Referent & 80.9\% & 24.2\%  \\
Clicks to Distractor & 19.1\% & 75.8\%  \\
\hline
\end{tabularx}
\end{table}


\subsection{Eye movements}
Figure \ref{fig:v2_eye} shows the time-course of fixations to referents and distractors over 2000~ms from referent onset, split by presence of cue.
Analyses conducted over the period from referent onset to 800~ms post-onset patterned with results from Experiment 1: 
When presented with an utterance unaccompanied by a visual cue, participants displayed a fixation bias towards the referent which increased over time (\resultsLM{3.03}{0.77}{=3.92}).
When presented with an utterance accompanied by an adaptor gesture, this bias was greatly reduced (\resultsLM{-3.00}{1.02}{=-2.95}).

\begin{figure}[Ht]
  \centering
	\includegraphics[width=\linewidth]{./img/e8_fixations.pdf}
  \caption{Eye-tracking results for Experiment 2: Proportion of fixations to each object (referent, distractor, video), from 0 to 2000 ms post-referent onset, calculated out of the total sum of fixations for each 20~ms time bin. Shaded areas represent $\pm$ 1 standard error of the mean.}
  \label{fig:v2_eye}
\end{figure}

\subsection{Mouse movements}
Figure \ref{fig:v2_mouse} shows the time-course of the proportions of cumulative distance the mouse moved towards the referent and distractor for 2000~ms from referent onset, split by presence of cue.
Analysis on the period from 0 to 800~ms post-referent onset patterned with the eye-tracking data:
Following no-cue videos, participants showed a tendency to move the mouse increasingly towards referent over this period (\resultsLM{1.90}{0.39}{=4.93}).
As with eye-movements, this referent-bias was greatly reduced following an adaptor gesture (\resultsLM{-2.46}{0.77}{=-3.19})

\begin{figure}[Ht]gesture
  \centering
	\includegraphics[width=\linewidth]{./img/e8_mouset.pdf}
  \caption{Mouse-tracking results for Experiment 2: Proportion of cumulative distance traveled toward each object from 0 to 2000 ms post-referent onset. Proportions were calculated from the total cumulative distance participants moved the mouse until that time bin (from speech-onset, when cursor was made visible). Shaded areas represent $\pm$ 1 standard error of the mean.}
  \label{fig:v2_mouse}
\end{figure}

\section{General discussion}
Participants' behaviour during the eye-tracking task in Experiment 2 patterned with the results from Experiment 1:
In both experiments, utterances presented with the speaker in a neutral posture and not gesturing biased listeners towards believing the speaker to be truthful, as shown by increased tendency to fixate on, move the mouse towards, and eventually click on the object which was named by the speaker. 
This result parallels the finding in \citet{Loy2017} where fluent (as opposed to disfluent) utterances saw a bias to infer the speaker to be truthful, and suggests that listeners may have an implicit honesty-bias when faced with no obvious potential cue to deception. 

Experiment 2 supports the findings from Experiment 1 that listeners associate adaptor gesturing with being deceitful.
The influence of visual cues on deception judgments was found to occur at the early stages of reference comprehension, with Experiment 2 finding that the initial bias towards the named-object over the distractor-object was completely attenuated when presented with an adaptor gesture. 
This suggests that the visual channel during the delivery of an utterance, just like the manner of spoken delivery, can modulate listeners' judgments about a speaker's intentions concurrently with the processing of lexical information.

This association between increased movement and dishonesty is consistent with previous research on beliefs about and judgments on visual cues to deception \citep{Zuckerman1981}.
The results from Experiment 1 show that the influence of the visual channel on judgments of an utterance's veracity extends to both the static and dynamic cues which accompany speech.

Interestingly, in both experiments, the bias towards the distractor --- signifying perceived dishonesty --- over the referent appeared approximately 1000~ms after the referent began, at a later point than previous versions of this paradigm in which speech disfluencies were found to modulate judgments of deception. 
This contrasts to a view of listeners relying on rule-of-thumb associations between visual cues and deception.
Our attempts to explore whether adaptor gestures are associated with deception via some more complex two-step reasoning about the perceived nervousness of the speaker were inconclusive: Participant's ratings of how nervous the speaker looked in each video did not pattern with their final judgments beyond whether or not the video presented an adaptor gesture, suggesting that the associations between these gestures and lying patterns with, but is not necessarily a consequence of perceived anxiety in the speaker.
To investigate further the mechanism underlying associations between nonverbal information and deception, it may be worth extending this notion to other options such as perceived effort of the speaker, something which we did not consider.%JK reword maybe.

Our results show that the integration of the visual channel can have a rapid and direct effect on pragmatic judgements, supporting the idea that communication is fundamentally multimodal: 
Speech and gesture interactively codetermine meaning.
However, the integration of visual cues to inform deception judgments appears to be more gradual than the integration of spoken cues.
To better understand how information in different modalities affect comprehension, further research would require investigating the effect of spoken delivery when the visual channel is also available --- for example, studying the time course of deception judgments when faced with one or both of a disfluency and an adaptor gesture.
%i.e. GVD 

\bibliography{./GCD}

\end{document}
